% !TeX root = ../main.tex
% Add the above to each chapter to make compiling the PDF easier in some editors.

\chapter{Introduction}\label{chapter:introduction}

The implemented low-level synchronization component in this master thesis is part of the KIA4SM project of the chair of operating systems. The work aims to implement a method for dynamic update of tasks ready queues in L4 Fiasco.OC/Genode while providing a synchronized access to them.


\section{Overview of KIA4SM project}

KIA4SM (stands for Cooperative Integration Architecture for Future Smart Mobility Solutions) is a research project at the department of operating systems \cite{kia4sm}. Traditionally Cooperative Intelligent Transport Systems(C-ITS) have been built on heterogeneous systems. The KIA4SM project aims to provide an architecture of having homogeneous software platform for heterogeneous hardware systems. The project focuses on developing systems for the interaction and coordination between
computer-assisted vehicles, be it partially or fully autonomously functioning actors and also aims to improve on the ad-hoc networking between vehicles

The final vision of the project is illustrate in figure \ref{kia4sm}. The goals of the project are,

\begin{figure}[h]
  \centering
  \includegraphics[scale = 1]{figures/kia4sm_vision.png}
  \caption{KIA4SM vision - homogeneous platform for heterogeneous devices \cite{kia4sm}} \label{kia4sm}
\end{figure}

\begin{itemize}
\item A common platform as foundation for device independent (vehicles, mobile devices, traffic and
transport architecture) provision and execution of software-based functionality
\item Mechanisms that allow for online dynamic reconfiguration, based on
\begin{itemize}
\item en-/disabling and relocation/migration of
software-based functionality
\item adaptive (data-centric) routing policy
\item flexible scheduling of tasks per ECU
\end{itemize}
\end{itemize}

In order to achieve the goals of the project a number of different methods have been applied. This has led to the application of Organic computing paradigm.
Organic Computing (OC) has the vision to address the
challenges of complex distributed systems by making them more
life-like (organic), i.e. endowing them with abilities such as self-
organization, self-configuration, self-repair, or adaptation.
%Refer: paper: Organic Computing – Addressing Complexity by Controlled Self-organization 
In order to realize this, universally applicable Electronic Control Units (ECU) and a common run-time environment are used which provides Hardware/Software Plug-and-Play properties.

\section{Motivation}

There are number of micro controllers used for different calculations in a modern vehicle, KIA4SM project aims to replace them with use of more power-full and standardized hardware, universally applicable ECUs. 

OC approach proposes a Observer-Controller architecture similar to MAPE architecture(monitor, analyze, plan, execute) . An observer collects the data from the all the ECUs and computes and generates indicators where a controller takes a decision based on the indicator and generates an action.

One such action of the controller is to decide what tasks should be executed at what time in order to meet the aforementioned requirements of the safety critical systems. It is essential to be able to add threads and modify the execution order during operation time.
A flexible thread handling is also required for example in case 
a ECU is malfunctioning. In this case it would be possible to swap the threads from the 
malfunctioning one to working ones. So it is important to generate new ready-queues 
based on the information we are receiving from the other ECUs in the grid, and then 
exchange them with the actual ready-queue the scheduler uses. 

The controller decides and produces a run/ready queue (RQ). There needs to be a method which allows to safely update the scheduler ready queue of the system. The work in this thesis concentrates on the scheduler ready queue update mechanism.

\begin{figure}[h][architecture]
  \centering
  \includegraphics[scale = 0.5]{figures/microkernel_architecture.png}
  \caption{Organic Computing: Applying the Observer/Controller pattern to existing microkernel architecture \cite{kia4sm}}\label{architeture}
\end{figure}

\section{Thesis structure}
The thesis is structured in a way that the reader understands the importance of the work carried out here and also the surrounding concepts before delving in to the specifics (inverted triangle). 

The second chapter summarizes the related work on the state of the art algorithms for synchronization and different types of schedulers in use and at the end of the section an evaluation of the synchronization methods is provided in order to chose the best possible approach for the existing project.

The third chapter explains the Genode and L4 Fiasco.OC details in brief, in order for the user to have an overview of the system. 

The fourth chapter deals with the design considerations along with implementation details.

The fifth chapter is dedicated to explains testing method and the results obtained.
And the final chapter concludes the thesis with the limitations and future work to be done. 