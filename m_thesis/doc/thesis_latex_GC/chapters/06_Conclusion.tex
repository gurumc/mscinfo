\chapter{Summary, Future Work and Conclusion}
This chapter summarizes the thesis, discusses the importance of the work and findings, outlines the limitations and indicates directions for future work. 

This chapter has been divided into four sections. Section \ref{summary} is a summary of the thesis. Thesis contributions and limitations are discussed in \ref{disc}. Section \ref{futurework} outlines the possible future work, and Section \ref{con} concludes the thesis.

\section{Summary of the Thesis} \label{summary}
This thesis introduced a method for updating the task ready queues in Genode and L4/Fiasco.OC system and the motivation for developing the Synchronization module. The Synchronization module is used in the Observer-Controller architecture.

Chapter 2 gave an overview of the related work. The work done by H{\"a}cker in his bachelor thesis served as a basis for current thesis work. There are lock-based and lock-free synchronization methods available and each type has certain benefits associated with them in terms of \textit{implementation, read and write speed, security, deadlocks}. The lock-free methods, Read-Copy Update(RCU) and Software Transactional Memory (STM) are better suited for this thesis work.

Chapter 3 discussed the concepts necessary to understand the thesis. Genode OS framework is a tool kit for building special purpose operating system which can be used to build safe, secure and robust operating systems. It has a recursive tree structure and each node of the tree is owned by its parent, which has access to its own sandbox. It can be used with different microkernels. L4/Fiasco.OC is the preferred choice of microkernel for KIA4Sm project, since it offers many features such as, paravirtualization, multi processor support etc. This is a object capability system, where everything in kernel is represented as an object and they interact with each other through a kernel provided IPC mechanism. The thread execution in kernel is divided between execution context, which takes care of execution parameters and scheduling context object, which holds scheduling parameters.

Chapter 4 and 5 explained the design and implementation details of the thesis. The communication between the Controller and Synchronization module is that of a producer-consumer relationship and the communication between these two modules is handled by shared dataspace, which is synchronized by using the locks provided from Genode. The limitations in accessing the ready queue from the user-level application, led to the development of kernel module. The application interacts with the Genode API, which intern calls the L4 layer. The L4 layer makes the flex pages for the threads being sent from Genode and send them to the scheduler kernel object with an IPC call. The scheduler decides the right time for updating ready queue and updates the threads. A new ready queue list can be created with the threads and exchange with the actual ready queue or existing ready queue can be updated.

In Chapter 6, the testing of the individual components is described. Since the integration testing was not possible the individual components are tested. The trace service of Genode is extended to return the Genode thread ids, which are sent to the kernel module to update them to the ready queue.

\section{Discussion} \label{disc}
\subsection{Thesis Contribution}
This thesis has developed a method for updating the kernel ready queue. A user-level component has been developed to handle the communication with the Controller. The developed kernel module handles the ready queue update mechanism. The thesis gives insights to the working environment of Fiasco.OC kernel and its interaction with Genode OS framework. The thread creation model in Genode and the thread migration in L4/Fiasco.OC is explained in details to understand the scheduling method of the L4/Fiasco.OC scheduler.

\subsection{Limitations}
The presented design in this thesis is a decent method to update the task ready queue of the scheduler. However, there exists a few limitations to the method presented. First the user-level application handles each ready queue one at a time and has to wait till it acquires the lock. This may lead to starvation of threads. The user-level Genode component cannot access the kernel ready queue list  directly and also it cannot make calls to L4 API layer of the kernel. This limits the scope of the user-level component. The thread update to the ready queue works well and the execution of the thread continues and creating a new ready queue list to exchange with the actual ready queue happens, however, the ready queue exchange mechanism doesn't ensure the thread execution. The current implementation of kernel module is limited to work with fixed priority scheduler.

\section{Future Work}\label{futurework}
This section describes the ideas for future research to be done on this concept. Further investigations should be done to identify the best approach to integrate the Controller and the Synchronization module. The proposed idea for integration is, the Observer gathers data including the Genode thread ids to the Controller and the Controller has to take decisions on scheduling and make the corresponding thread ids available in the shared dataspace. The user-level Synchronization module can read this dataspace and update them to ready queue using kernel module. And also the kernel module extension is required to make it work with the different types of schedulers.

The initial aim of the thesis was to develop a high-level component for ready queue update. However, this had to be changed since the ready queue was not accessible from a high-level component. Further work should be carried out find out if a user-level component can replace the kernel module's work. The communication between user-level module to kernel module involves many API calls, further research to be done on this regard to reduce the number of API calls.

\section{Conclusion}\label{con}
The Synchronization module developed in this thesis is a decent method for updating the threads to kernel ready queue. It serves as a good starting point for the \texttt{Observer-Controller} architecture's goal of having a user-level component for ready queue update. However, this module can be extended to suite the overall goal of the KIA4SM project.